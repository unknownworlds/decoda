\documentstyle[a4,makeidx,verbatim,texhelp,fancyhea,mysober,mytitle]{report}%
\usepackage{url}
\date{March 2009}
\parskip=10pt%
\parindent=0pt%
\title{Decoda}%
\makeindex%
\begin{document}%
\maketitle%
\pagestyle{fancyplain}%
\bibliographystyle{plain}%
\pagenumbering{roman}%
\setheader{{\it CONTENTS}}{}{}{}{}{{\it CONTENTS}}%
\setfooter{\thepage}{}{}{}{}{\thepage}%
\tableofcontents%

\chapter{Introduction}\label{intro}
\pagenumbering{arabic}%
\setheader{{\it CHAPTER \thechapter}}{}{}{}{}{{\it CHAPTER \thechapter}}%
\setfooter{\thepage}{}{}{}{}{\thepage}%

Decoda is a professional development environment for debugging Lua script in your applications. It's familiar and fast and you'll wonder how you ever worked without it.
\index{licensing}
\index{support}
\index{Features}

\textbf{Features}
\begin{itemize}
\item Full graphical IDE. Syntax highlighting, symbol browsing/filtering, configurable hotkeys and colors, custom tools and more. If you know MSVC, you'll be comfortable immediately (F5 to run, F9 for breakpoints, etc).
\item Plays well with others. Integrates with your source control (SCC) and is simple enough for the un-nerdly. It even attaches to MSVC for debugging your native code.
\item Plug and play. It requires no code changes and supports Lua 4.0 and later. It supports the latest and greatest games and will work with your creation too.
\end{itemize}

Decoda is Copyright (c) 2007-2011, Unknown Worlds Entertainment, Inc.

\chapter{Getting Started}\label{getting_started}
\index{breakpoints}

\textbf{Quick start}
\begin{itemize}
\item Add all your Lua files (Project -> Add Existing File). Shift-select all your files and press Open.
\item Point Decoda at your .exe (Project -> Settings). Enter your project's .exe, command-line arguments (windowed mode if needed) and the working directory and press OK.
\item Save your project (Project -> Save Project As).
\item Press Control-Q to jump to the Project Explorer and type the name of a class to jump to, or browse to it via the tree view on the left. Press F9 to set a breakpoint somewhere.
\item Press F5 to run. Your application should start and hit your breakpoint. If you get warnings in the output window, see our \helpref{Warnings section}{warnings}.
\end{itemize}

\textbf{Other notes}
\begin{itemize}
\item Decoda is designed to work just like the debugger in Microsoft Visual Studio. F9 to set breakpoints, F5 to run, F10 to step over, F11 to step in, etc.  
\item If you don't have the Lua source files (ie, you're debugging a commercial application that uses Lua), Decoda will still show you code and symbols it loads.
\item Follow \helpref{these steps}{source_control} to integrate Decoda with your source control (SCC).
\end{itemize}

\chapter{Registering}\label{registering}

\textbf{30-Day Evaluation License}

Any individual person is entitled to use Decoda for 30 days for the purposes of
evaluating the software. This 30-day period starts on the day that the user first
installs the product and ends 30 days thereafter.

When the 30 days has lapsed, Decoda will cease functioning. If the user wishes
to continue using Decoda past this date, the user will have to register his or her copy.

\textbf{Registration License}

Decoda is licensed on a per-user basis.  Each individual person who uses Decoda
beyond the 30-day evaluation period will need his or her own license.

If the user uses more than one PC, he or she only requires a single license. If two or
more users share a PC, each user will require his or her own license.

\textbf{Pricing}

Please visit the Decoda web site for the latest pricing information, at
\urlref{http://www.unknownworlds.com/decoda}{http://www.unknownworlds.com/decoda}. 

\textbf{How to Register}

You first need to obtain a registration key. Details on obtaining the key can be found
at \urlref{http://www.unknownworlds.com/decoda}{http://www.unknownworlds.com/decoda}.
Once you have obtained a key, enter it in the start up screen.  The key can also be
entered while the application is running by selecting Help/Register from the menu. 

\chapter{Debugging}\label{debugging}

\textbf{Interface tips}
\begin{itemize}
\item Select a variable and drag it to the Watch window or type in something more specific.
\item You can debug your native code by checking the Attach to System Debugger option and then pressing F5.
\item Double-click in the Call Stack window while debugging to see your code context.
\end{itemize}

\section{Debugging a New Process}\label{debugging_new}

The Debug/Start menu option initiates debugging of a new process.  Settings for the debugging
session, such as the executable to run, are configured in the project settings. If the project
settings have not been configured when starting a debugging session, the Project Settings dialog
will automatically be displayed. See the section on \helprefn{Project Settings}{project_settings}
for more information.

Once the debugging session begins, the application should start and messages will be printed to
the Output window. The output during a typical debugging session will look something like this:

{\verbatim Debugging session started
Debugger attached to process
0x00150fd0: VM created
... }

The message "Debugger attached to process" indicates that the debugger has finished its
initialization and is ready to debug. Initialization is not finalized until the application being
debugged begins to access the Lua API, so they message may not be display right away.

If you don't see this message during a debugging session, check for warnings that may have been
printed to the output window.  Consult the section on \helprefn{warnings}{warnings} for further
assistance.

\section{Attaching to a Running Process}\label{attaching_debugger}

Decoda can debug a process that is already running by using the Debug/Processes... menu option.
Selecting this option will display a list of all the processes currently running on the system.
Clicking on a process and pressing the Attach button will.

Note, the debugger cannot be attached to a process that is currently or has at any point during
its lifetime been debugged by Decoda. This means that if you debug a process, detach from it, and
then attempt to reattach to the process it will fail.

Due to the nature of Decoda's interaction with Lua, there are some limitations when attaching to a
running process.  Specifically, the debugger will not have full information on events that occurred
before it was attached to the process.  These events include the creation of virtual machines or
the loading of scripts.

When attaching to a running process, Decoda will use the symbol directory specified in the current
project settings to search for PDB files. See the section on \helprefn{Project Settings}{project_settings}
for more information.

\section{Displaying userdata in the Watch window}\label{watch}
\index{userdata}
\index{__towatch}
\index{__tostring}

By default, Decoda does not have enough information about userdata values to display detailed information
in the Watch window.  Your application can alter this behavior by supplying additional hooks using either the 
__towatch or __tostring metamethods.

To display the value of a userdata in the Watch window, Decoda will first try to invoke the __towatch method
on the userdata's metatable. This method will be invoked with the userdata as a parameter and should return a
value to be used in place of the userdata in the Watch window.  The most common use is to return a table which
contains keys and values for the member elements of the userdata.

The example code below shows how a __towatch function may be written for a simple Point structure.

{\verbatim struct Point
\{
	float x;
	float y;
\};

int PointToWatch(lua_State* L)
\{

	const Point* point = (const Point*)(lua_touserdata(L, 1));

	lua_newtable(L);
	int table = lua_gettop(L);
	
	lua_pushstring(L, "x");
	lua_pushnumber(L, point->x);
	lua_settable(L, table);

	lua_pushstring(L, "y");
	lua_pushnumber(L, point->y);
	lua_settable(L, table);
	
	return 1;
	
\} }


If the userdata does not have a metatable, or the table does not contain a __towatch method, Decoda will try
to use the __tostring metamethod to get a textual description of the userdata.  This method will be invoked
with the userdata as a parameter and should return a string description of the object.  More information about
__tostring can be found in the Lua documentation.

\section{Using a Modified Version of Lua}\label{modified_lua}

Decoda utilizes the Lua implementation of the application being debugged and therefore can work with modified
versions of Lua. The only restriction is that the modified version must retain the original Lua API functions.
Decoda expects functions to have the same prototype as in the original API and generally behave similarly (have
the same return values and side effects).

If you wish to modify the Lua API and still maintain compatibility with Decoda, the correct way is to create
functions with new names that utilize the original API functions.

\textbf{Example:}

{\verbatim int luaL_newmetatablewithtype(lua_State *L, const char *tname, int type) \{
  
    int result = luaL_newmetatable(L, tname);

    lua_pushnumber(L, type);
    lua_setfield(L, -2, "Type");

    return result;

\} }

In this example, where ever you were using luaL_newmetatable call this new luaL_newmetatablewithtype.

\section{Simultaneously Debugging C/C++ and Lua}

A normal C/C++ debugger such as Microsoft Visual Studio can be used alongside Decoda to
provide complete debugging of an application.

This can be accomplished by debugging a new process (see \helprefn{Debugging a New Process}{debugging_new})
from Decoda and then using the C/C++ debugger to attach to the running process.  Conversely, the process can
be started from inside the C/C++ debugger and then attached from Decoda (see
\helprefn{Attaching to a Running Process}{attaching_debugger}).

If the Debug/Attach to System Debugger menu option is checked, Decoda will automatically invoke the system
debugger whenever a new process is started. The system debugger is specified in the Windows registry and will
typically be set to Microsoft Visual Studio if it is installed on the machine. This mechanism provides a shortcut
for attaching both a C/C++ debugger and Decoda to an application simultaneously.

\chapter{User Interface}\label{user_interface}
\pagenumbering{arabic}%
\setheader{{\it CHAPTER \thechapter}}{}{}{}{}{{\it CHAPTER \thechapter}}%
\setfooter{\thepage}{}{}{}{}{\thepage}%

\textbf{Tips}
\begin{itemize}
\item Press Control-Q and type a class or function name to find it quickly.
\item By default, Decoda acts like MSVC (F5 to debug, F9 to toggle breakpoints, etc.)
\item You can customize your IDE colors (Tools -> Settings).
\item Add a Lua compiler to your Tools menu to quickly check for errors.
\end{itemize}

\section{Project Settings}\label{project_settings}

The Project Settings dialog is accessed from the Project/Settings menu and is used to
setup information about the application being debugged.

\textbf{Debugging Options}

\begin{twocollist}
\twocolitemruled{{\bf Field}}{{\bf Description}}
\twocolitem{ Command }{ The full path to the executable file that will be launched with the debugging session }
\twocolitem{ Command Arguments }{ Any command line arguments that will be passed to the application }
\twocolitem{ Working Directory }{ The default directory for the process being debugged. If this field is left blank the executable's directory is used as the working directory. }
\twocolitem{ Symbols Directory }{ The directory to search for PDB files. Multiple directories separated by semi-colons can also be specified. }
\end{twocollist}

\textbf{Source Control Options}

The source control fields in the Project Settings dialog always you to setup your project
to connect to an existing source control project.  The Source Control Provider field lists
the registered SCC plugins installed on the computer, and the Project field lets you connect
to a source control project.

The exact user interface provided by the source control system is dependent on the SCC plugin
that is being used. Please check your SCC plugin documentation for more information.

More information about source control integration can be found in the section on \helprefn{Source Control}{source_control}.

\section{Virtual Machines}\label{virtual_machines}

Virtual machines correspond to lua_State objects and are created when lua_newstate
or other similar functions -- such as lua_open and luaL_newstate -- are called.  All of the virtual
machines that currently exist in the program being debugged are shown the Virtual Machines
window.

By default, virtual machines are identified in the UI by their address, however they
can be named via the Decoda \helprefn{API}{decoda_name} for additional clarity.

Selecting a VM updates the watch window

\section{Project Explorer}\label{project_explorer}

\index{Project Explorer}

The Project Explorer window is used to browse the files, functions and classes in
the current project.

During a debugging session, any scripts loaded by the debugee will automatically
be added to the Project Explorer window if they aren't already in the project. These
files are Temporary files are indicated by a faint "ghost file" icon and will be removed
when the debugging session ends.  If the files are open in the editor at the end of the
debugging session they'll be removed from the Project Explorer when they're closed.

Items displayed in the Project Explorer can be filtered by entering text in the box
at the top of the window. Only files and symbols that begin with the specified text
string will be displayed. Adding a space to the beginning of your filter will
shows symbols that contain the text \em{anywhere} in the name.  The input focus
can be moved to the Project Explorer filter text box by using the Window/Open In Filter
menu option. Using a hotkey with this menu item (Ctrl+Q by default) makes it very
easy to rapidly locate functions and files using only the keyboard.

 \image{5cm;0cm}{images/project_explorer_filter}

The files and symbols in the Project Explorer can also be filtered by the properties
of the files. This is accomplished by using the file filter drop down menu located next
to the text box at the top of the window. The filter can be used to show or hide files
that are Temporary, Unversioned, Checked In or Checked Out. Files that have any of the
checked properties will be displayed in the Project Explorer window. So for instance,
to hide the Temporary files added during the debugging session, uncheck the Temporary
icon and leave the others checked.

\section{Source Control}\label{source_control}
\pagenumbering{arabic}%
\setheader{{\it CHAPTER \thechapter}}{}{}{}{}{{\it CHAPTER \thechapter}}%
\setfooter{\thepage}{}{}{}{}{\thepage}%

\index{Source Control}
\index{SCC}
\index{Perforce}
\index{CVS}
\index{Subversion}
\index{SVN}

Decoda includes support for Microsoft SCC plugins.  SCC plugins are available for most popular
source control systems including Microsoft Visual SourceSafe, Perforce, CVS and Subversion (SVN) and
can be obtained from the vendor or other third parties.

Decoda will automatically scan the Window's registry for SCC plugins, so no special steps are
required beyond following the vendor's instructions for installing the plugin on your system.

Source control functions are accessed through the SCC menu or by right clicking on items in the
Project Explorer window.

\section{External Tools}\label{external_tools}

\index{External Tools}
\index{\$(ItemPath)}
\index{\$(ItemDir)}
\index{\$(ItemFileName)}
\index{\$(ItemExt)}

The External Tools dialog enables you to customized the Tools menu with new items that
invoke external applications.  An example use is to setup the Lua compiler (luac.exe) to
parse the document you are editing to check for syntax errors.

Macros can be inserted into the Arguments field to substitute information about the active
document when the tools is invoked. The table below summarizes the available macros. 

\begin{twocollist}
\twocolitemruled{{\bf Macro}}{{\bf Description}}
\twocolitem{ \$(ItemPath) }{ Complete directory, file name and extension. }
\twocolitem{ \$(ItemDir) }{ Directory the file is in (no trailing backslash) }
\twocolitem{ \$(ItemFileName) }{ Name of the file with no directory }
\twocolitem{ \$(ItemExt) }{ File extension (no preceding period) }
\end{twocollist}

Like all of the menu items, items added through the External Tools dialog can be assigned to
hot keys using the Key Bindings dialog.

\section{System Settings}\label{system_settings}

\begin{itemize}
\item File associations allows you to select files that will be opened when double clicking
on them in Windows explorer.
\item Check for updates with automatically when the application starts. Checking for updates
does not send any information (personal or otherwise).
\end{itemize}

\chapter{Command Line Switches}\label{command_line}
\pagenumbering{arabic}%
\setheader{{\it CHAPTER \thechapter}}{}{}{}{}{{\it CHAPTER \thechapter}}%
\setfooter{\thepage}{}{}{}{}{\thepage}%

\index{command line switches}
\index{arguments}

\begin{tabular}{|l|p{8.5cm}|}\hline
\row{{\bf Command line switch}&{\bf Description}}\hline\hline
\row{\helprefn{/debugexe}{debugexe} filename&Starts debugging of the specified executable file.}
\end{tabular}

Files can be specified on the command line.

\section*{/debugexe filename}\label{debugexe}
\index{/debugexe}

Starts debugging of the specified executable file. If the file name contains spaces, it should
be enclosed in quotes.

\textbf{Example:}

Decoda.exe /debugexe "c:$\backslash$Program Files$\backslash$Application$\backslash$Application.exe"

If a project is also loaded on the command line and the executable image for the project
matches the file specified with /debugexe, the working directory and arguments specified
in the project will be used when launching the executable. Otherwise the directory of the
executable will be used as the working directory and no command line arguments will be
supplied.

\chapter{Warnings}\label{warnings}

This section describes warnings that may be written to the Output window while debugging. While
warnings do not necessarily mean there is an error, they often point to unexpected behavior that
may cause problems or unexpected behavior.

\textbf{Warnings}

\section{Warning 1000: Lua functions were not found during debugging session}\label{warning_1000}
\index{Warning 1000}

This warning is displayed when the debugger fails to locate the necessary Lua functions inside
the main executable or any of the DLLs loaded during the debugging session. If the application
statically links with Lua, this may indicate that the necessary PDB information could not be
found by the debugger. This can also occur if the application is missing some of the necessary
Lua functions or they have been renamed.

This warning can also occur under normal circumstances if the application was terminated before
it loaded the DLL containing the Lua functions.

\section{Warning 1001: '...' appears to contain Lua functions however no Lua functions could located with the symbolic information}\label{warning_1001}
\index{Warning 1001}

This warning means that a module of the application being debugged contains a binary signature
that indicates the presence of the Lua implementation.  The PDB for the module, however, was
either missing or did not include information about the Lua functions.

To address this problem, you may need to adjust the Symbol Directory in the
\helprefn{Project Settings}{project_settings}.

\section{Warning 1002: Symbol file '...' located but it does not match module '...'}\label{warning_1002}
\index{Warning 1002}

This warning means that during the process of debugging, a PDB which does not match the
corresponding module was encountered. The most common cause of this problem is that the module
is built in Release mode, while the PDB is for the Debug mode version of the code.

If the named module contains the Lua implementation, then this warning may result in the debugger
being unable to locate the necessary Lua functions. If this happens the debugger functions will
not work properly.

This warning may also be displayed if the format of the PDB prevents it from being loaded, either
because it is an obsolete format or is not actually in the PDB format.

\section{Warning 1003: Not all virtual machines were destroyed}\label{warning_1003}
\index{Warning 1003}

This warning means that the application being debugged created virtual machines that were not
destroyed before the application exited. For example, if the application uses Lua 5.0, not all
lua_newstate calls had corresponding lua_close calls.

\section{Warning 1004: Couldn't hook Lua function '...'}\label{warning_1004}
\index{Warning 1004}

This warning means that during the hooking of Lua functions during the debugging process,
the specified function was not found. This can be the result of the Lua function missing
from the Lua implementation due to modifications, or because the symbolic information for
that function is missing.

This warning can also occur if the application being debugged statically links with Lua and is
compiled with optimizations enabled. In this case, certain Lua functions that are not used
may be removed by the compiler during the optimization phase.  One way to fix this problem is
to add an explicit call the Lua function named in the warning and recompile the application.

Alternatively, if you are using the Microsoft Visual C++ linker you can use the /OPT:NOREF
option to prevent the linker from removing these functions.

\section{Warning 1005: lua_call/lua_pcall called with too few arguments on the stack}\label{warning_1005}
\index{Warning 1005}
\index{lua_call}
\index{lua_pcall}

This warning indicates that the lua_call or lua_pcall Lua API function was called from C
with fewer than the expected number of arguments on the Lua stack.  These functions both
expect there to be one plus the number of arguments the function accepts on the stack.

Please consult the Lua API documentation for more information.

\section{Warning 1006: lua_pushthread could not be emulated due to modifications to Lua. Coroutines may be unstable}\label{warning_1006}
\index{Warning 1006}
\index{lua_pushthread}
\index{coroutine}
\index{thread}

Decoda requires the lua_pushthread function to properly handle coroutines. This
function was introduced in Lua 5.1 and doesn't exist in the Lua 5.0 API. To
work around this, Decoda will attempt to emulate the functionality when debugging
an application that uses Lua 5.0.

This emulation can fail if the application is using a heavily modified version of
Lua 5.0. To work around this, you can add an implementation of lua_pushthread to
your Lua 5.0 code. To do this, add the following code to lapi.c:

{\verbatim LUA_API int lua_pushthread (lua_State *L) \{
  lua_lock(L);
  setthvalue(L->top, L);
  api_incr_top(L);
  lua_unlock(L);
  return (G(L)->mainthread == L);
\}
}

If you do not supply an implementation of lua_pushthread and it cannot be emulated,
the use of coroutines may cause Decoda to become unstable.

\section{Warning 1007: Just-in-time compilation of Lua code disabled to allow debugging}\label{warning_1007}
\index{Warning 1007}
\index{LuaJIT}
\index{JIT}
\index{thread}

The just-in-time compilation provided by LuaJIT is incompatible with the mechanisms that Decoda
uses for debugging. In order to allow debugging, Decoda automatically disables just-in-time
compilation for any Lua code that is loaded.

This should have no side effects other than possibly causing your application to execute more slowly,
since Lua code will be interpreted rather than compiled. This warning does not indicate a problem
and can be safely ignored.

\section{Warning 1008: No __tostring or __towatch metamethod was provided for userdata}\label{warning_1008}
\index{Warning 1008}
\index{__tostring}
\index{__towatch}
\index{userdata}

A userdata value that was displayed in the Watch window did not have a __tostring or __towatch function
in its metatable.  Without one of these functions Decoda is unable to provide detailed information about
the value of the userdata. Please refer to \helprefn{this}{watch} article for more information.

\section{Warning 1009: Enabling LuaJIT C call return work-around}\label{warning_1009}
\index{Warning 1009}
\index{__tostring}
\index{__towatch}
\index{userdata}

LuaJIT version 2.0 does not call the debug hook callback with LUA_HOOKCALLRET when returning from a C
function call. Normally Decoda uses this during debugging. If Decoda detects that LuaJIT 2.0 is being
used a work-around is used. This warning does not indicate a problem and can be safely ignored.

\chapter{API}\label{api}

Decoda exposes an API through Lua that can be used to control aspects of the debugger. The API is
implemented as a set of Lua functions that are automatically registered into your Lua states when
they are created under the debugger.  The API can either be accessed directly from Lua or from C.

The API won't exist if the debugger is not running, so it's important to check that the desired
function exists before attempting to call it.  This example demonstrates how to call an API function
from Lua:

{\verbatim if decoda_output then decoda_output("Test") }

The same function can be called from C like this:

{\verbatim const char* message = "Test";
lua_getglobal(L, "decoda_output");

if (lua_isnil(L, -1))
\{
    lua_pop(L, 1);
\}
else
\{
    lua_pushstring(L, message);
    lua_call(L, 1, 0);
\}
}

\textbf{API Functions}

\section{decoda_name}\label{decoda_name}

Sets the name of the virtual machine displayed in the debugger. This is not a function, but
a variable that should be set by the application being debugged to alter the name.

\textbf{Example:}

{\verbatim lua_pushstring(L, "Client VM");
lua_setglobal(L, "decoda_name"); }

\section{decoda_output(message)}\label{decoda_output}

Writes a text message to the output window of the debugger.

\textbf{Example:}

{\verbatim const char* message = "This is a test message";
lua_getglobal(L, "decoda_output");

if (lua_isnil(L, -1))
\{
    lua_pop(L, 1);
\}
else
\{
    lua_pushstring(L, message);
    lua_call(L, 1, 0);
\}
}

\chapter{Contact}\label{contact}
\index{contact}

If you experience bugs or are having difficulty to getting your application
to work with Decoda, feel free to contact us at our technical support e-mail
address:

\urlref{decoda@unknownworlds.com}{mailto:decoda@unknownworlds.com}

Please provide details about your issue and the version number, which can be
found in the Help/About dialog. We'll be happy to assist you with both the
registered and unregistered versions of Decoda.

\chapter{License Agreement}\label{license}

You should carefully read the following terms and conditions before using Decoda.
Unless you have a different license agreement signed by Unknown Worlds Entertainment,
your use of this software indicates your acceptance of this license agreement and
warranty. If you do not accept these terms you must cease using this software immediately.

\textbf{Copyright}

Decoda ("the Software") is copyright (c) 2007-2011 by Unknown Worlds Entertainment, Inc., all
rights reserved.

\textbf{Evaluation and Registration}

This is not free software.  You are hereby licensed to use the Software for evaluation
purposes without charge for a period of 30 days.  If you use the Software after the 30-day
evaluation period a registration fee as described in the documentation is required.
Unregistered use of the Software after the 30-day evaluation period is in violation of
U.S. and international copyright laws.

One registered copy of the Software may either be used by a single person who uses the
software personally on one or more computers, or installed on a single computer used by
multiple people, but not both.

An organization may purchase a multi-user site license amounting to either the number of
individuals who have access to the Software on any number of computers, or the number of
computers that will access the Software by any number of individuals.  A site is defined
as all company locations within a 100 mile radius. Site licenses do not cover subsidiary
companies, or customers of a company.

\textbf{Distribution}

Provided that you do not include your License Key you are hereby licensed to make copies
of the Software; give exact copies of the original to anyone; and distribute the Software
in its unmodified form via electronic means.  You are specifically prohibited from charging
for any such copies.

\textbf{Governing Law}

This agreement shall be governed by the laws of the State of California.

\textbf{Disclaimer of Warranty}

THIS SOFTWARE AND THE ACCOMPANYING FILES ARE SOLD "AS IS" AND WITHOUT WARRANTIES AS TO
PERFORMANCE OR MERCHANTABILITY OR ANY OTHER WARRANTIES WHETHER EXPRESSED OR IMPLIED.  NO
WARRANTY OF FITNESS FOR A PARTICULAR PURPOSE IS OFFERED.  ANY LIABILITY OF THE SELLER WILL
BE LIMITED EXCLUSIVELY TO REFUND OF PURCHASE PRICE.

\end{document}
